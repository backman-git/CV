% LaTeX file for resume 
% This file uses the resume document class (res.cls)

\documentclass{res} 
\usepackage{hyperref}
%\usepackage{helvetica} % uses helvetica postscript font (download helvetica.sty)
%\usepackage{newcent}   % uses new century schoolbook postscript font 
\setlength{\textheight}{9.5in} % increase text height to fit on 1-page 

	\begin{document} 
		% Center the name over the entire width of resume:
		\moveleft.5\hoffset\centerline{\large\bf CHUNG YI KAO}
		% Draw a horizontal line the whole width of resume:
		\moveleft\hoffset\vbox{\hrule width\resumewidth height 1pt}\smallskip
		% address begins here
		% Again, the address lines must be centered over entire width of resume:
		\moveleft.5\hoffset\centerline{Hsinchu Taiwan}
		\moveleft.5\hoffset\centerline{(+886) 933216219}
		\moveleft.5\hoffset\centerline{backman.only@gmail.com}
																
		\begin{resume}
 			\vspace{-0.2in}
			\section{EDUCATION}
			\vspace{0.1in}
				{\sl Master of Computer Science}, National Tsing Hua University, Taiwan. \hfill Sept. 2015\\
				{\sl B​achelor of Computer Science}, ​National Chung Cheng University, Taiwan. \hfill Sept. 2013\\ 

 			\vspace{-0.2in}	
			\section{EXPERIENCE}
	 			\vspace{-0.05in}	
				\begin{tabbing}
	 				\hspace{2.3in}\= \hspace{2.6in}\= \kill % set up two tab positions
					{\bf Senior Software Engineer} \>Mstar Semiconductor     \>Jan. 2016 - Sept. 2016\\
														 \>Hsinchu, Taiwan
	 			\end{tabbing}\vspace{-20pt}      % suppress blank line after tabbing
	 			\vspace{0.2in}	
				\begin{itemize}
					\item Android smart TV division
					\vspace{-0.05in}	
					\item Implement image loader for product line. (Reactive programming)
					\vspace{-0.05in}	
					\item Auto translation program for multi-language string table (speed up development)
					\vspace{-0.05in}	
					\item SHARP Smart TV Project
					\vspace{-0.05in}	
					\item enhance terminal development

				\end{itemize}
				\begin{tabbing}
	 				\hspace{2.3in}\= \hspace{2.6in}\= \kill % set up two tab positions
					{\bf iOS Development Engineer} \>Industrial Technology Research Institute	
					\>July 2013 - Sept. 2015\\
													\>Hsinchu, Taiwan
	 			\end{tabbing}\vspace{-20pt}
	 			\vspace{0.2in}	
				\begin{itemize}
					\item Develop client side App(iOS) of Smart Cold Chain Logistics System\footnote{​{\bf Paper}  \url{http://www.ece.uci.edu/~chou/pdf/chou-ithings14container2.pdf}}.
					\vspace{-0.05in}	
					\item Design a lightweight algorithm of detection about status of container on embedded platform.
				\end{itemize}
				\begin{tabbing}%
					\hspace{2.3in}\= \hspace{2.6in}\= \kill % set up two tab positions          
	 				{\bf Android Development Engineer }  \>KerjaDulu\> Nov. 2014 - July 2015\\
													\>Taipei, Taiwan
				\end{tabbing}\vspace{-20pt}
				\vspace{0.2in}	
				\begin{itemize}
					\item Develop KerjaDulu App: SQL Lite, UI effect
				\end{itemize}

			
			\section{TECHNICAL SKILLS}          
				\begin{itemize}
					\item {\bf Program Language}: Python, Java, JavaScript, C++
					\vspace{-0.05in}
					\item {\bf Framework}: Node.js, Jade, Less.js, PhoneGap, Matplotlib, SQL, MongoDB, OpenCV
					\vspace{-0.05in}
					\item {\bf Software Engineering}: UML, Design Pattern, OOP
					\vspace{-0.05in}
					\item {\bf Math}: probability, machine learning	
					\vspace{-0.05in}
					\item {\bf English}: TOEIC 840 pts.
				\end{itemize}

			\section{HONORS AND AWARDS}          
				Won {\bf Q​ualcomm award} in Mobileheroes\footnote{{\bf MobileHeros}  \url{https://www.mobilehero.com/component/k2/item/209-a2012_11-10.html?Itemid=196}}, {\sl B​ook Finder}\footnote{{\bf Book Finder} \url{https://www.youtube.com/watch?v=BvLKtrgq_yw}   }  \hfill 2012\\
				W​on {\bf Excellent work} in ICT Innovative Services\footnote{{\bf ICT}  \url{http://innoserve.tca.org.tw/en/index.aspx}}, {\sl T​ouch Projector​}\footnote{{\bf Touch Projector}  \url{https://www.youtube.com/watch?v=nt4eUAXqXyk}} \hfill 2012\\  
			\vspace{-0.08in}	
			\section{PROJECT}
			\begin{itemize}
					\item EcoSim: A Smartphone-Based Sensor-Node Simulator with Native Sensor and Protocol-Stack Emulation\footnote{{\bf EcoSim} \url{https://www.youtube.com/watch?v=1UUcqf0pjM0}}
					\vspace{-0.05in}
					\item Gesture recognition with IoT platform based on triaxial acceleration\footnote{{\bf Gesture recognition} \url{https://www.youtube.com/watch?v=VInyJABrmPo}}  \hfill2​015
					\vspace{-0.05in}
					\item use Machine Learning to hack CAPTCHA \footnote{{\bf Hack CAPTCHA} \url{https://www.youtube.com/watch?v=9ovWzIu1zy8}}
					\vspace{-0.05in}
					\item Voice Recognition and Correction \footnote{{\bf Voice Recognition and Correction} \url{https://www.youtube.com/watch?v=xZo-bpWrYlk}}					
			\end{itemize}
				

				          
			 
		\end{resume}
	\end{document}