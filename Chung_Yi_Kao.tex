% LaTeX file for resume
% This file uses the resume document class (res.cls)

\documentclass{res}
\usepackage{helvet}
\usepackage{enumitem}
\usepackage[colorlinks = true,
            linkcolor = blue,
            urlcolor  = blue,
            citecolor = blue,
            anchorcolor = blue]{hyperref}

\makeatletter
\renewcommand{\section}{\@startsection{section}{1}{0mm}
  {-\baselineskip}{0.5\baselineskip}{\bf\leftline}}
\makeatother


\usepackage[top=1in,left=0.5in,bottom=0in]{geometry}

	\begin{document}

		\voffset=-1.5cm
		\moveleft.5\hoffset\centerline{\large\bf CHUNG YI KAO}
		% Draw a horizontal line the whole width of resume:
		\moveleft\hoffset\vbox{\hrule width\resumewidth height 1pt}\smallskip
		\moveleft.5\hoffset\centerline{backman.only@gmail.com \href{https://github.com/backman-git}{https://github.com/backman-git} }


		\begin{resume}

 			\vspace{-0.3in}
 			\section{SUMMARY}
 			\vspace{-0.25in}
 				A technophile with the passion of using computer science knowledge to improve the quality of work and life.
 			\vspace{-0.1in}
 			\section{SKILL}
 			\vspace{-0.1in}
 			\begin{itemize}
					\item Python, C, Java, PHP, JavaScript
					\vspace{-0.05in}
					\item Laravel, PostgreSQL, Tensorflow, E.L.K, Docker, Vagrant, kubernetes
					\vspace{-0.05in}
					\item GDB, Xdebug, Valgrind, Web Cache, JVM, Embedded System
					\vspace{-0.05in}
					\item Simulator(EcoSim), Interpreter(PHP.js), VM(JVM.py).
			\end{itemize}
			\vspace{-0.1in}
			\section{EXPERIENCE}
			\vspace{-0.2in}


%Rakuten
				{\bf Rakuten}			 {\hfill Aug. 2018 - Current}\\
				{\bf Service Application Engineer}   {\hfill Tokyo}
				\begin{itemize}
					\item Developed new features of service.
					\vspace{-0.05in}
					\item Improved Development Enviroment and Tools.
					\vspace{-0.05in}
					\item Migrated the service to Laravel framework.
				\end{itemize}
				\vspace{-0.10in}

% mandatory service
				{\bf Mandatory Service}			 {\hfill Sept. 2016 - July 2017}\\
				{\bf Emergency Services Command Center Engineer}   {\hfill Yulin, Taiwan}
				\begin{itemize}
					\item Implemented a human resource assistant ERP system and login App.(3 days to 1 day)
					\\  This is SPA client side render smart leave system, I use Redux.js and React.js to implement. Express.js to implement backend system and PostgreSQL to store Data. I also develop an interesting UX experience call  "cell as stamp". It uses the QRCode as a method to identify personal ID(oAuth2) and make the usage of the system easier and more intuitive.
				\end{itemize}
				\vspace{-0.10in}


%Mstar
				{\bf Mstar Semiconductor} 		 {\hfill Jan. 2016 - Sept. 2016}\\
				{\bf Senior Software Engineer}		 {\hfill Hsinchu, Taiwan}

				\begin{itemize}
					\item Cooperate with Amazon software team in smart TV project)
					\vspace{-0.05in}
					\item Developed Android system stack for smart TV (Binder, service)
					\vspace{-0.05in}
					\item Implemented electronic program guide application.
					\vspace{-0.05in}
					\item Implemented an automatic tool for multi-language translation \\(Improved developing efficiency: 5hs to 1s)
					\vspace{-0.05in}
					\item Implemented the logger.(FileBeat)
					\vspace{-0.05in}
					\item UML code injection.(Runtime sequence diagram for debugging)
				\end{itemize}
				\vspace{-0.10in}
% ITRI
				{\bf Industrial Technology Research Institute } {\hfill July 2013 - Sept. 2015}\\
				{\bf Fullstack Development Engineer}                        {\hfill Hsinchu, Taiwan}\\
				Smart Insulating Container with Anti-Theft Features by M2M Tracking {\footnotesize in Proc. 2014 IEEE (iThings 2014)}
				\vspace{0.05in}
				\begin{itemize}
					\item Implemented iOS monitor application, logging system, backend system for a cold-chain logistics system.
					\vspace{-0.05in}
					\item Recognition algorithm for anti-theft.
				\end{itemize}
				\vspace{-0.10in}

%kerjadulu
	 			{\bf KerjaDulu }                                {\hfill  Nov. 2014 - July 2015}\\
				{\bf Android Development Engineer}			    	 {\hfill Taipei, Taiwan}

				\begin{itemize}
					\item {\href{https://www.techinasia.com/kerjadulu-funding-mnc-group}{KerjaDulu Human Resource App}}: KerjaDulu is a human resource app which targets Indonesia labor market.
					\vspace{-0.05in}
					\item Native android framework to build an android app.
					\vspace{-0.05in}
					\item Node.js and PostgreSQL to build the backend system.
					\vspace{-0.05in}
					\item E.L.K: Data Visualization.
				\end{itemize}
				\vspace{-0.10in}

%librarian
				{\bf Campus Library}                            {\hfill  Nov. 2010 - July 2013}\\
				{\bf MIS engineer of library services}            {\hfill Chaiyi, Taiwan}
				\begin{itemize}
					\item Implemented B​ook Finder Android App to locate books
				\end{itemize}

			\section{HONORS AND AWARDS}
				\begin{itemize}
					\item {\bf IEEE Conference Paper} in \href{https://expo.itri.org.tw/2019VLSIDAT/Program/SessionView/Regular}{VLSI-DAT}, \href{https://hdl.handle.net/11296/x3c4a6}{\sl EcoSim: Sensor-Node Simulator} \hfill 2019
					\vspace{-0.05in}
					\item {\bf Q​ualcomm Award} in \href{https://www.mobilehero.com/component/k2/item/209-a2012_11-10.html?Itemid=196}{MobileHeros}, \href{https://www.youtube.com/watch?v=BvLKtrgq_yw}{\sl B​ook Finder} \hfill 2012
					\vspace{-0.05in}
					\item {\bf Excellent Work} in \href{http://innoserve.tca.org.tw/en/index.aspx}{ICT} Innovative Services, \href{https://www.youtube.com/watch?v=nt4eUAXqXyk}{\sl T​ouch Projector​} \hfill 2012
				\end{itemize}

			\section{EDUCATION}
			\vspace{-0.20in}
				{\sl Master of Computer Science}, National Tsing Hua University, Taiwan. \hfill Sept. 2015\\
				{\footnotesize Thesis: EcoSim: A Smartphone-Based Sensor-Node Simulator with Native Sensor and Protocol-Stack Emulation}





			\section{PROJECT}
			\vspace{-0.1in}
			\begin{itemize}

					\item EcoSim: A Smartphone-Based Sensor-Node Simulator with Native Sensor and Protocol-Stack Emulation
					\vspace{-0.05in}

					\item \href{https://github.com/backman-git/Interpreter.js}{\sl PHP.js}
					\vspace{-0.05in}

					\item \href{https://github.com/backman-git/JVM.py}{\sl JVM.py}
					\vspace{-0.05in}

					\item Using Machine Learning to hack CAPTCHA
					\vspace{-0.05in}

					\item Voice Recognition and Correction (ED algorithm)
					\vspace{-0.05in}

					\item Smart Leave System (Alternative Military Service)(API server, AWS, oAuth2)
					\vspace{-0.05in}

					\item Touch projector
					\vspace{-0.05in}

					\item Book Finder (Indoor location service of library)
					\vspace{-0.05in}

					\item Gesture recognition with IoT platform based on triaxial acceleration
					\vspace{-0.05in}

			\end{itemize}


		\end{resume}
	\end{document}