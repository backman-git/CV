% LaTeX file for resume 
% This file uses the resume document class (res.cls)

\documentclass{res} 
\usepackage{helvet}
\usepackage{enumitem}
\usepackage[colorlinks = true,
            linkcolor = blue,
            urlcolor  = blue,
            citecolor = blue,
            anchorcolor = blue]{hyperref}

\makeatletter 
\renewcommand{\section}{\@startsection{section}{1}{0mm}
  {-\baselineskip}{0.5\baselineskip}{\bf\leftline}}
\makeatother


\usepackage[top=1in,left=0.5in,bottom=0in]{geometry}

	\begin{document} 

		\voffset=-1.5cm
		\moveleft.5\hoffset\centerline{\large\bf CHUNG YI KAO}
		% Draw a horizontal line the whole width of resume:
		\moveleft\hoffset\vbox{\hrule width\resumewidth height 1pt}\smallskip
		\moveleft.5\hoffset\centerline{backman.only@gmail.com \href{https://github.com/backman-git}{https://github.com/backman-git} }
	

		\begin{resume}
			

 			\vspace{-0.3in}
 			\section{SUMMARY} 
 			\vspace{-0.25in}
 				A technophile with the passion of using computer science knowledge to improve the quality of work and life.
 			\vspace{-0.1in}
 			\section{SKILL}
 			\vspace{-0.1in}
 			\begin{itemize}[leftmargin=*]
					\item iOS and Android app development: Java ,Objective-C , 
					\vspace{-0.05in}
					\item web development:AWS, Restful api server,OAuth2 ,Node.js ,JavaScript ,postgreSQL ,MongoDB ,Redux.js ,React.js ,express.js, webpack.js 		
					\vspace{-0.05in}
					\item Keras machine learning framework:Python , E.L.K
					
			\end{itemize}
			\vspace{-0.1in}
			\section{EXPERIENCE}
			\vspace{-0.2in}

% mandatory service

				{\bf Mandatory Service}			 {\hfill Sept. 2016 - July 2017}\\
				{\bf Emergency Services Command Center Engineer}   {\hfill Yulin, Taiwan}
				\begin{itemize}[leftmargin=*]
					\vspace{-0.05in}
					\item Implemented a human resource assistant ERP system and login App.(3 days to 1 day) 
					\\  This is SPA client side render smart leave system, I use Redux.js and React.js to implement. Express.js to implement backend system and PostgreSQL to store Data. I also develop an interesting UX experience call  "cell as stamp". It uses the QRCode as a method to identify personal ID(oAuth2) and make the usage of the system easier and more intuitive.





				\end{itemize}
				\vspace{-0.12in}
%Mstar
				{\bf Mstar Semiconductor} 		 {\hfill Jan. 2016 - Sept. 2016}\\
				{\bf Senior Software Engineer}		 {\hfill Hsinchu, Taiwan}

				\begin{itemize}[leftmargin=*]

					\item cooperate with Amazon software team in smart TV project)
					\vspace{-0.05in}	
					\item Developed Android system stack for smart TV (Binder, service)	
					\vspace{-0.05in}
					\item Implemented electronic program guide application.
					\vspace{-0.05in}
					\item Implemented an automatic tool for multi-language translation \\(Improved developing efficiency: 5hs to 1s)
					\vspace{-0.05in}		
					\item Implemented logger.(Increased debugging efficiency from 10mins to 5s)
					\vspace{-0.05in}	
					\item UML code injection.(Runtime sequence diagram for debugging)

				\end{itemize}
				\vspace{-0.12in}
% ITRI
				{\bf Industrial Technology Research Institute } {\hfill July 2013 - Sept. 2015}\\	
				{\bf iOS Development Engineer (part-time)}                        {\hfill Hsinchu, Taiwan}\\
				Smart Insulating Container with Anti-Theft Features by M2M Tracking {\footnotesize in Proc. 2014 IEEE (iThings 2014)}
				\vspace{0.05in}
				\begin{itemize}[leftmargin=*]


					\item Implemented iOS monitor application, logging system, backend system for a cold-chain logistics system.
					\vspace{-0.05in} 
					\item Recognition algorithm for anti-theft.
				\end{itemize}
				\vspace{-0.12in}

%kerjadulu	    
	 			{\bf KerjaDulu }                                {\hfill  Nov. 2014 - July 2015}\\
				{\bf Android Development Engineer(part-time)}			    	 {\hfill Taipei, Taiwan}
		
				\begin{itemize}[leftmargin=*]
					\item {\href{https://play.google.com/store/apps/details?id=com.kerjadulu.kerjadulu&hl=zh_TW}{KerjaDulu Human Resource App}}: Project Starter, OOP Design, UI effect, SQLite, RESTful API, postgreSQL.
				\end{itemize}
				\vspace{-0.12in}

%librarian
				{\bf Campus Library}                            {\hfill  Nov. 2010 - July 2013}\\
				{\bf MIS engineer of library services (part-time)}            {\hfill Chaiyi, Taiwan}
				\begin{itemize}[leftmargin=*]
					\item Implemented B​ook Finder Android App to locate books
					\vspace{-0.05in} 
					\item Implemented assistant tools of library services
					\vspace{-0.05in}
					\item Maintained Library Services and System server
				\end{itemize}

			\vspace{-0.1in}
			\section{HONORS AND AWARDS}  
			    \vspace{-0.25in}
				\noindent {\bf Q​ualcomm award} in \href{https://www.mobilehero.com/component/k2/item/209-a2012_11-10.html?Itemid=196}{MobileHeros}, \href{https://www.youtube.com/watch?v=BvLKtrgq_yw}{\sl B​ook Finder} \hfill 2012\\
				{\bf Excellent work} in \href{http://innoserve.tca.org.tw/en/index.aspx}{ICT} Innovative Services, \href{https://www.youtube.com/watch?v=nt4eUAXqXyk}{\sl T​ouch Projector​} \hfill 2012\\  
	
			\vspace{-0.25in}
			\section{EDUCATION}
			\vspace{-0.25in}
				{\sl Master of Computer Science}, National Tsing Hua University, Taiwan. \hfill Sept. 2015\\
				{\footnotesize Thesis: EcoSim: A Smartphone-Based Sensor-Node Simulator with Native Sensor and Protocol-Stack Emulation}

			\vspace{-0.1in}
			\section{PROJECT}
			\vspace{-0.1in}
			\begin{itemize}[leftmargin=*]

					\item Smart Leave System (Alternative Military Service)(API server, AWS, oAuth2) 
					\vspace{-0.05in}
					
					\item EcoSim: A Smartphone-Based Sensor-Node Simulator with Native Sensor and Protocol-Stack Emulation
					\vspace{-0.05in}

					\item Touch projector
					\vspace{-0.05in}
					
					\item Book Finder (Indoor location service of library)
					\vspace{-0.05in}
					
					\item Gesture recognition with IoT platform based on triaxial acceleration
					\vspace{-0.05in}
					
					\item Using Machine Learning to hack CAPTCHA	
					\vspace{-0.05in}
					
					\item Voice Recognition and Correction (ED algorithm)
					\vspace{-0.05in}


			\end{itemize}
			
				          
		\end{resume}
	\end{document}